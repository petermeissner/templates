% // Citation //
% Thanks to Martin Puppe !
% Source: https://gist.github.com/puppe/4378686
%

\documentclass[%
  fontsize=12pt, % Schriftgröße
%  version=last%  % Neueste Version von KOMA-Skript verwenden
]{scrlttr2}

% ===== Deutsche Sprache =====
\usepackage[utf8]{inputenc}
\usepackage[ngerman]{babel}
% ============================

\LoadLetterOption{DIN} % Einstellungen für DIN 676 laden

\KOMAoptions{%
% fromemail=true,       % Email wird im Briefkopf angezeigt
% fromphone=true,       % Telefonnumer wird im Briefkopf angezeigt
% fromfax=true,         % Faxnummer wird im Briefkopf angezeit
% fromurl=true,         % URL wird im Briefkopf angezeigt
% fromlogo=true,        % Logo wird im Briefkopf angezeigt
% subject=titled,       % Druckt "Betrifft: " vor dem Betreff
locfield=wide,          % Breite Absenderergänzung (location)
fromalign=left,         % Ausrichtung des Briefkopfes
fromrule=afteraddress%  % Trennlinie unter dem Briefkopf
}

\RequirePackage[utf8]{inputenc}
\RequirePackage[ngerman]{babel}

\setkomavar{fromname}{Max Mustermann} % Name
\setkomavar{fromaddress}{% % Adresse
  Musterstr. 42\\
  01234 Musterstadt%
}
\setkomavar{fromfax}{01234~56789} % Faxnummer
\setkomavar{fromemail}{max.muster@muster.com} % Email-Adresse
\setkomavar{fromphone}{01234~56789} % Telefonnummer
\setkomavar{fromurl}[Website:~]{www.muster.com} % Website

\firsthead{} 

% ===== Absenderergänzung =====
\setkomavar{location}{%
  \raggedright\footnotesize{%
  \usekomavar{fromname}\\
  \usekomavar{fromaddress}\\
    \vspace{0.5em} 
  \usekomavar*{fromphone}\usekomavar{fromphone}\\
  %\usekomavar*{fromfax}\usekomavar{fromfax}\\
  \usekomavar*{fromemail}\usekomavar{fromemail}
  \usekomavar*{fromurl}\usekomavar{fromurl}}%
}
% ============================

% Logo
% \setkomavar{fromlogo}{\includegraphics{logo.png}}

% Die Bankverbindung wird nicht automatisch verwendet. Dazu muss bspw. mittels \firstfoot ein eigener Brieffuß definiert werden.
\setkomavar{frombank}{}

% ===== Signatur =====
\setkomavar{signature}{%
  \usekomavar{fromname}\\
  Geschäftsführer%
}
\renewcommand*{\raggedsignature}{\raggedright}
% ====================

\usepackage{graphicx} % Um Grafiken (bspw. das Logo) einbinden zu können

\begin{document}

\begin{letter}{%
% ===== Zielanschrift =====
  Erika Musterfrau\\
  Musterweg 43\\
  56789 Musterhausen%
% =======================
}

% ====== Geschäftszeichenzeile =========
\setkomavar{yourref}{}          % Ihr Zeichen
\setkomavar{yourmail}{}         % Ihr Schreiben vom
\setkomavar{myref}{}            % Unser Zeichen
\setkomavar{customer}{}         % Kundennummer
\setkomavar{invoice}{}          % Rechnungsnummer
\setkomavar{place}{Musterstadt} % Ort
\setkomavar{date}{\today}       % Datum
% =====================================

\setkomavar{title}{}
\setkomavar{subject}{Betreff}

\opening{Sehr geehrte Frau Musterfrau}

hier kommt der Text hin.

% ===== Postskriptum =====
%\ps PS: \dots
% ========================

% ===== Anlage(n) =====
% \setkomavar*{enclseparator}{Anlage}
\encl{%
  Anlage 1\\
  Anlage 2%
}
% ===================

% ===== Verteiler =====
% \setkomavar*{ccseparator}{Kopie an}
%\cc{Verteiler 1, Verteiler 2}
% =====================

\end{letter}
\end{document}